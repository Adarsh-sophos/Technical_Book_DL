\usepackage[T1]{fontenc}
\usepackage{bm}
\usepackage{bbm}
\usepackage[utf8]{inputenc}
\usepackage{latexsym}
\usepackage[english]{babel}
\usepackage{indentfirst}
%\usepackage{fullpage}
\usepackage{graphicx}
\usepackage{lmodern}
%\usepackage{epsfig}
%\usepackage[math]{anttor}
\usepackage[sc]{mathpazo}
%\usepackage{fouriernc}
%\usepackage[garamond]{mathdesign}
\usepackage{geometry}
\input Kramer.fd
\usepackage{yfonts,color}
\usepackage{minitoc}
\usepackage{titlesec}
\usepackage{subfigure}
\usepackage{amsmath}
\usepackage{amssymb}
\usepackage{stmaryrd}
\usepackage{url}
\usepackage{pgfplots}
\pgfplotsset{compat=1.5}
\usepackage[colorlinks,linkcolor=red!80!black,
citecolor=red!80!black,pdfpagelabels,hyperindex=true]{hyperref}
\hypersetup{
%    bookmarks=true,         % show bookmarks bar?
%    unicode=false,          % non-Latin characters in Acrobat’s bookmarks
%    pdftoolbar=true,        % show Acrobat’s toolbar?
%    pdfmenubar=true,        % show Acrobat’s menu?
%    pdffitwindow=false,     % window fit to page when opened
%    pdfstartview={FitH},    % fits the width of the page to the window
%    pdftitle={My title},    % title
%    pdfauthor={Author},     % author
%    pdfsubject={Subject},   % subject of the document
%    pdfcreator={Creator},   % creator of the document
%    pdfproducer={Producer}, % producer of the document
%    pdfkeywords={keyword1} {key2} {key3}, % list of keywords
%    pdfnewwindow=true,      % links in new window
    colorlinks=true,       % false: boxed links; true: colored links
    linkcolor=brown!80!black,  % color of internal links (change box color with linkbordercolor)
    citecolor=green!50!black,        % color of links to bibliography
    filecolor=magenta,      % color of file links
    urlcolor=red!80!black           % color of external links
}
\usepackage{csquotes}
\usepackage[sorting=none,backend=bibtex,style=numeric-comp]{biblatex}
\usepackage{cancel}
%\usepackage{natbib}
\usepackage{pifont}
%\bibliographystyle{plain}
%\bibliographystyle{unsrt}
\usepackage{tikz}
\usetikzlibrary{matrix,arrows,decorations,backgrounds,shapes,calc,fit}
\usetikzlibrary{fadings}
\usetikzlibrary{decorations.pathmorphing}
\usetikzlibrary{decorations.markings}
\usepackage{xcolor}
\usepackage{amsfonts}
\usepackage{slashed}
\usepackage{fancybox}
\usepackage{fancyhdr}
\newenvironment {abstract}%
{\cleardoublepage \null \vfill \begin{center}%
\bfseries \abstractname\end{center}}%
{\vfill \null}
\newcommand{\intkaha}{\ensuremath{\int\frac{\d^3\ka}{4|\ka||p-k|(2\pi)^3}}}
\newcommand{\slv}{\raise.15ex\hbox{$/$}\kern-.53em\hbox{$v$}}
\newcommand{\slF}{\raise.15ex\hbox{$/$}\kern-.53em\hbox{$F$}}
\newcommand{\slL}{\raise.15ex\hbox{$/$}\kern-.53em\hbox{$L$}}
\newcommand{\slP}{\raise.15ex\hbox{$/$}\kern-.53em\hbox{$P$}}
\newcommand{\slp}{\raise.15ex\hbox{$/$}\kern-.53em\hbox{$p$}}
\newcommand{\slq}{\raise.15ex\hbox{$/$}\kern-.53em\hbox{$q$}}
\newcommand{\slR}{\raise.15ex\hbox{$/$}\kern-.53em\hbox{$R$}}
\newcommand{\slQ}{\raise.15ex\hbox{$/$}\kern-.53em\hbox{$Q$}}
\newcommand{\slK}{\raise.15ex\hbox{$/$}\kern-.53em\hbox{$K$}}
\newcommand{\slk}{\raise.15ex\hbox{$/$}\kern-.53em\hbox{$k$}}
\newcommand{\slD}{\raise.15ex\hbox{$/$}\kern-.73em\hbox{$D$}}
\newcommand{\slC}{\raise.15ex\hbox{$/$}\kern-.53em\hbox{$C$}}
\newcommand{\slA}{\raise.15ex\hbox{$/$}\kern-.53em\hbox{$A$}}
\newcommand{\slSigma}{\raise.15ex\hbox{$/$}\kern-.53em\hbox{$\Sigma$}}
\newcommand{\slpartial}{\raise.15ex\hbox{$/$}\kern-.53em\hbox{$\partial$}}
\newcommand{\slcalP}{\raise.15ex\hbox{$/$}\kern-.63em\hbox{$\cal P$}}
\definecolor{purp}{RGB}{0,0,0}
\def\p{{\boldsymbol p}}
\def\P{{\boldsymbol P}}
\def\q{{\boldsymbol q}}
\def\Q{{\boldsymbol Q}}
\def\l{{\boldsymbol l}}
\def\k{{\boldsymbol k}}
\def\kp{{\boldsymbol k}_{{\tiny \perp}}}
\def\m{{\boldsymbol m}}
\def\x{{\boldsymbol x}}
\def\xp{{\boldsymbol x}_{{\tiny \perp}}}
\def\yp{{\boldsymbol y}_{{\tiny \perp}}}
\def\pp{{\boldsymbol p}_{{\tiny \perp}}}
\def\y{{\boldsymbol y}}
\def\X{{\boldsymbol X}}
\def\Y{{\boldsymbol Y}}
\def\D{{\boldsymbol D}}
\def\r{{\boldsymbol r}}
\def\z{{\boldsymbol z}}
\def\v{{\boldsymbol v}}
\def\w{{\boldsymbol w}}
\def\b{{\boldsymbol b}}
\def\u{{\boldsymbol u}}
\newcommand{\intkk}{\ensuremath{\int\frac{\d^3\ka}{2|\ka|(2\pi)^3}}}
\renewcommand{\d}{\ensuremath{\mathrm{d}}}
\newcommand{\old}{\ensuremath{\text{old}}}
\newcommand{\niou}{\ensuremath{\text{new}}}
\newcommand{\nab}{\ensuremath{\text{\boldmath$\nabla$}}}
\newcommand{\ix}{\ensuremath{\text{\boldmath $x$}}}
\newcommand{\igrec}{\boldsymbol{y}}
\newcommand{\ixl}{\ensuremath{\text{\scriptsize\boldmath $x$}}}
\newcommand{\ka}{\ensuremath{\text{\boldmath $k$}}}
\newcommand{\parti}{\ensuremath{\text{\boldmath $\partial$}}}
\newcommand{\uu}{\ensuremath{\text{\boldmath $u$}}}
\newcommand{\vv}{\ensuremath{\text{\boldmath $v$}}}
\newcommand{\pel}{\ensuremath{\mbox{\scriptsize\boldmath $p$}}}
\newcommand{\uul}{\ensuremath{\text{\scriptsize\boldmath $u$}}}
\newcommand{\vvl}{\ensuremath{\text{\scriptsize\boldmath $v$}}}
\newcommand{\kal}{\ensuremath{\text{\scriptsize\boldmath $k$}}}
\newcommand{\el}{\ensuremath{\text{\scriptsize\boldmath $l$}}}
\newcommand{\cigma}{\ensuremath{\text{\boldmath$\Sigma$}}}
\newcommand{\Sig}{\ensuremath{\mbox{\boldmath $\Sigma$}}}
\newcommand{\Sigl}{\ensuremath{\mbox{\scriptsize\boldmath $\Sigma$}}}
\newcommand{\pe}{\ensuremath{\mbox{\boldmath $p$}}}
\newcommand{\ine}{\ensuremath{\mathrm{in}}}
\newcommand{\oute}{\ensuremath{\mathrm{out}}}
\newcommand{\intk}{\ensuremath{\int\frac{\d^3\ka}{2|\k|(2\pi)^3}}}
\newcommand{\intmod}[1]{\ensuremath{\int\frac{\d^3{\bm #1}}{2|{\bm #1}|(2\pi)^3}}}
\newcommand{\intmodd}[2]{\ensuremath{\iint\frac{\d^3{\bm #1}\,\d^3{\bm #2}}{4|{\bm #1}||{\bm #2}|(2\pi)^6}}}
\newcommand{\ma}[1]{{\mathcal{#1}}}
\newcommand{\SK}{ \textsc{Schwinger-Keldysh} }
\newcommand{\Fnman}{ \textsc{Feynman} }
\newcommand{\pointe}[2]{
\node[place5] at (#1,#2) {};
\node[place4,xshift=rand*0.3mm,yshift=rand*0.3mm] at (#1,#2) {};	}
\newcommand{\pointee}[2]{
\node[place,rotate around={45:(0,0)},xshift=rand*0.4mm,yshift=rand*2.75mm] at (#1,#2) {};
\node[place2,rotate around={45:(0,0)},xshift=rand*0.4mm,yshift=rand*2.75mm] at (#1,#2) {};
\node[place3,rotate around={45:(0,0)},xshift=rand*0.4mm,yshift=rand*2.75mm] at (#1,#2) {};}
\newcommand{\pointeee}[2]{
\node[place2,rotate around={135:(0,0)},xshift=rand*0.4mm,yshift=rand*2.75mm] at (#1,#2) {};
\node[place3,rotate around={135:(0,0)},xshift=rand*0.4mm,yshift=rand*2.75mm] at (#1,#2) {};
\node[place,rotate around={135:(0,0)},xshift=rand*0.4mm,yshift=rand*2.75mm] at (#1,#2) {};	}
\newcommand{\point}[2]{
\node[place,xshift=rand*1.5mm,yshift=rand*12.8mm] at (#1,#2) {};
\node[place2,xshift=rand*1.5mm,yshift=rand*12.8mm] at (#1,#2) {};
\node[place3,xshift=rand*1.5mm,yshift=rand*12.8mm] at (#1,#2) {};}
\newcommand{\pointt}[2]{
\node[place2,xshift=rand*1.5mm,yshift=rand*12.8mm] at (#1,#2) {};
\node[place3,xshift=rand*1.5mm,yshift=rand*12.8mm] at (#1,#2) {};
\node[place,xshift=rand*1.5mm,yshift=rand*12.8mm] at (#1,#2) {};}
\tikzstyle{blueballa} = [circle,shading=ball, ball color=blue!30,inner sep =1.2mm]
\tikzstyle{redballa} = [circle,shading=ball, ball color=red,inner sep =1.2mm]
\tikzstyle{greenballa} = [circle,shading=ball, ball color=green!70!black,inner sep =1.2mm]
\tikzstyle{blueball} = [circle,shading=ball, ball color=blue!30,inner sep =0.3mm]
\tikzstyle{redball} = [circle,shading=ball, ball color=red,inner sep =0.3mm]
\tikzstyle{greenball} = [circle,shading=ball, ball color=green!70!black,inner sep =0.3mm]
\tikzstyle{gluon} = [thick, style={decorate,decoration={coil,amplitude=4pt, segment length=4pt}}]
\newcommand{\gluebr}[4]{
\node[blueball,xshift=rand*0.2cm,yshift=rand*1.2cm] (toto)at (#1,#2) {};
\node[redball,xshift=rand*0.2cm,yshift=rand*1.2cm] (tata) at (#3,#4) {};
\draw[photon] (toto) --(tata);}
\newcommand{\gluerb}[4]{
\node[redball,xshift=rand*0.2cm,yshift=rand*1.2cm] (toto)at (#1,#2) {};
\node[blueball,xshift=rand*0.2cm,yshift=rand*1.2cm] (tata) at (#3,#4) {};
\draw[photon] (toto) --(tata);}
\newcommand{\gluebg}[4]{
\node[blueball,xshift=rand*0.2cm,yshift=rand*1.2cm] (toto)at (#1,#2) {};
\node[greenball,xshift=rand*0.2cm,yshift=rand*1.2cm] (tata) at (#3,#4) {};
\draw[photon] (toto) --(tata);}
\newcommand{\gluegb}[4]{
\node[greenball,xshift=rand*0.2cm,yshift=rand*1.2cm] (toto)at (#1,#2) {};
\node[blueball,xshift=rand*0.2cm,yshift=rand*1.2cm] (tata) at (#3,#4) {};
\draw[photon] (toto) --(tata);}
\newcommand{\gluegr}[4]{
\node[greenball,xshift=rand*0.2cm,yshift=rand*1.2cm] (toto)at (#1,#2) {};
\node[redball,xshift=rand*0.2cm,yshift=rand*1.2cm] (tata) at (#3,#4) {};
\draw[photon] (toto) --(tata);}
\newcommand{\gluerg}[4]{
\node[redball,xshift=rand*0.2cm,yshift=rand*1.2cm] (toto)at (#1,#2) {};
\node[greenball,xshift=rand*0.2cm,yshift=rand*1.2cm] (tata) at (#3,#4) {};
\draw[photon] (toto) --(tata);}
\newcommand{\gluebra}[4]{
\node[blueball,xshift=rand*0.2cm,yshift=rand*1.8cm] (toto)at (#1,#2) {};
\node[redball,xshift=rand*0.2cm,yshift=rand*1.8cm] (tata) at (#3,#4) {};
\draw[photon] (toto) --(tata);}
\newcommand{\gluerba}[4]{
\node[redball,xshift=rand*0.2cm,yshift=rand*1.8cm] (toto)at (#1,#2) {};
\node[blueball,xshift=rand*0.2cm,yshift=rand*1.8cm] (tata) at (#3,#4) {};
\draw[photon] (toto) --(tata);}
\newcommand{\gluebga}[4]{
\node[blueball,xshift=rand*0.2cm,yshift=rand*1.8cm] (toto)at (#1,#2) {};
\node[greenball,xshift=rand*0.2cm,yshift=rand*1.8cm] (tata) at (#3,#4) {};
\draw[photon] (toto) --(tata);}
\newcommand{\gluegba}[4]{
\node[greenball,xshift=rand*0.2cm,yshift=rand*1.8cm] (toto)at (#1,#2) {};
\node[blueball,xshift=rand*0.2cm,yshift=rand*1.8cm] (tata) at (#3,#4) {};
\draw[photon] (toto) --(tata);}
\newcommand{\gluegra}[4]{
\node[greenball,xshift=rand*0.2cm,yshift=rand*1.8cm] (toto)at (#1,#2) {};
\node[redball,xshift=rand*0.2cm,yshift=rand*1.8cm] (tata) at (#3,#4) {};
\draw[photon] (toto) --(tata);}
\newcommand{\gluerga}[4]{
\node[redball,xshift=rand*0.2cm,yshift=rand*1.8cm] (toto)at (#1,#2) {};
\node[greenball,xshift=rand*0.2cm,yshift=rand*1.8cm] (tata) at (#3,#4) {};
\draw[photon] (toto) --(tata);}
\newcommand{\freezew}[2]{
\node[wball,xshift=rand*1.7cm,yshift=rand*1.4cm] at (#1,#2) {};}
\newcommand{\freezeb}[2]{
\node[bball,xshift=rand*1.7cm,yshift=rand*1.4cm] at (#1,#2) {};}
\tikzstyle{photon} = [very thin, style={decorate, decoration={snake,amplitude=0.4pt, segment length=2pt}}]
\tikzfading [name=radialfade, inner color=transparent!40, outer color=transparent!100]
\newcommand{\col}[1]{{\color{black} #1}}
\newcommand{\intp}{\int \frac{\d^2 \p }{(2\pi)^2}}
\newlength{\longueurAdHoc}
\settodepth{\longueurAdHoc}{$\displaystyle\int\limits_{y^->0}$}
\newcommand{\esss}{\begin{tikzpicture}[]
\draw[red!60!white,thick] (-0.06,-0.06) circle (0.12);
\draw[red!60!black,thick] (-0.06,-0.06) -- ++(45:0.12);
\draw[red!60!black,thick] (-0.06,-0.06) -- ++(135:0.12);
\draw[red!60!black,thick] (-0.06,-0.06) -- ++(315:0.12);
\draw[red!60!black,thick] (-0.06,-0.06) -- ++(225:0.12);
\end{tikzpicture}}
\newcommand{\matprodu}{\begin{tikzpicture}[]
\draw[red!60!white,thick] (0,0) circle (0.12);
\draw[red!60!black,thick] (0,0) -- ++(45:0.12);
\draw[red!60!black,thick] (0,0) -- ++(135:0.12);
\draw[red!60!black,thick] (0,0) -- ++(315:0.12);
\draw[red!60!black,thick] (0,0) -- ++(225:0.12);
\end{tikzpicture}}
\newcommand{\matprodp}{\begin{tikzpicture}[]
\draw[green!60!white,thick] (0,0) circle (0.12);
\draw[green!60!black,thick] (0,0) -- ++(45:0.12);
\draw[green!60!black,thick] (0,0) -- ++(135:0.12);
\draw[green!60!black,thick] (0,0) -- ++(315:0.12);
\draw[green!60!black,thick] (0,0) -- ++(225:0.12);
\end{tikzpicture}}
\newcommand{\matprode}{\begin{tikzpicture}[]
\draw[blue!60!white,thick] (0,0) circle (0.12);
\draw[blue!60!black,thick] (0,0) -- ++(45:0.12);
\draw[blue!60!black,thick] (0,0) -- ++(135:0.12);
\draw[blue!60!black,thick] (0,0) -- ++(315:0.12);
\draw[blue!60!black,thick] (0,0) -- ++(225:0.12);
\end{tikzpicture}}
\newcommand{\circa}[1]{\begin{tikzpicture}[baseline=-0.65ex]
\draw[thick] (0,0) node[black] {$#1$} circle (0.17);
\end{tikzpicture}}
\newcommand{\circb}[2]{\begin{tikzpicture}[baseline=(current bounding box.center)]
\draw[thick] (0,0) node[black] {$#1$} circle (0.17);
\node[anchor=north west] at (0.085,0) {\tiny $#2$};
\end{tikzpicture}}
\newcommand{\ells}{\begin{tikzpicture}[]
\draw[red!60!white,thick] (0,0) circle (0.12);
\draw[red!60!black,thick] (0,0) -- ++(45:0.12);
\draw[red!60!black,thick] (0,0) -- ++(135:0.12);
\draw[red!60!black,thick] (0,0) -- ++(315:0.12);
\draw[red!60!black,thick] (0,0) -- ++(225:0.12);
\end{tikzpicture}}
\newcommand{\etts}{\begin{tikzpicture}[]
\draw[green!60!white,thick] (0,0) circle (0.12);
\draw[green!60!black,thick] (0,0) -- ++(45:0.12);
\draw[green!60!black,thick] (0,0) -- ++(135:0.12);
\draw[green!60!black,thick] (0,0) -- ++(315:0.12);
\draw[green!60!black,thick] (0,0) -- ++(225:0.12);
\end{tikzpicture}}
\newcommand{\umumnud}[2]{\begin{tikzpicture}[every node/.style={sloped,allow upside down},baseline=0ex,place/.style={inner sep =0.2mm,circle,draw=black,fill=black}]
\draw[] (0,0)  -- node {\midarrow} (0.5,0)   --node {\midarrow}  (0.5,0.5) -- node {\midarrow} (0,0.5)node[anchor=south east] {$x$} --node {\midarrow}  (0,0);
\node[place] at (0,0.5) {};
\node[anchor=north] at (0.25,0) {\tiny $\hat{#1}$};
\node[anchor=east] at (0,0.25) {\tiny $\hat{#2}$};
\end{tikzpicture}}
\newcommand{\umumnu}[2]{\begin{tikzpicture}[every node/.style={sloped,allow upside down},baseline=0ex,place/.style={inner sep =0.2mm,circle,draw=black,fill=black}]
\draw[] (0,0) -- node {\midarrow} (0,0.5)  node[anchor=south east] {$x$}  --node {\midarrow}  (0.5,0.5) -- node {\midarrow} (0.5,0)--node {\midarrow}  (0,0);
\node[place] at (0,0.5) {};
\node[anchor=south] at (0.25,0.5) {\tiny $\hat{#1}$};
\node[anchor=west] at (0.5,0.25) {\tiny $\hat{#2}$};
\end{tikzpicture}}
\newcommand{\umunud}[2]{\begin{tikzpicture}[every node/.style={sloped,allow upside down},baseline=0ex,place/.style={inner sep =0.2mm,circle,draw=black,fill=black}]
\draw[] (0,0) node[anchor=north east] {$x$} -- node {\midarrow} (0,0.5)   --node {\midarrow}  (0.5,0.5) -- node {\midarrow} (0.5,0)--node {\midarrow}  (0,0);
\node[place] at (0,0) {};
\node[anchor=south] at (0.25,0.5) {\tiny $\hat{#1}$};
\node[anchor=east] at (0,0.25) {\tiny $\hat{#2}$};
\end{tikzpicture}}
\newcommand{\umunu}[2]{\begin{tikzpicture}[every node/.style={sloped,allow upside down},baseline=0ex,place/.style={inner sep =0.2mm,circle,draw=black,fill=black}]
\draw[] (0,0) node[anchor=north east] {$x$} -- node {\midarrow} (0.5,0)   --node {\midarrow}  (0.5,0.5) -- node {\midarrow} (0,0.5)--node {\midarrow}  (0,0);
\node[place] at (0,0) {};
\node[anchor=north] at (0.25,0) {\tiny $\hat{#1}$};
\node[anchor=west] at (0.5,0.25) {\tiny $\hat{#2}$};
\end{tikzpicture}}
\newcommand{\Emu}[1]{\begin{tikzpicture}[baseline=-0.5ex,place3/.style={inner sep =0.4mm,circle,draw=black,fill=green!70!black},]
\node[place3] at (0,0)  {};
\node[anchor=south] at (0,0) {$x$};
\node[anchor=north] at (0,0) {\tiny $#1$};
\end{tikzpicture}}
\newcommand{\emu}[1]{\begin{tikzpicture}[baseline=-0.5ex,place3/.style={inner sep =0.4mm,circle,draw=black,fill=blue!70!white},]
\node[place3] at (0,0)  {};
\node[anchor=south] at (0,0) {$x$};
\node[anchor=north] at (0,0) {\tiny $#1$};
\end{tikzpicture}}
\newcommand{\emup}[1]{\begin{tikzpicture}[baseline=-0.5ex]
\shade [ball color=blue!70!white] (0,0) circle [radius=0.075];
\node[anchor=south] at (0,0) {$x$};
\node[anchor=north] at (0,0) {\tiny $#1$};
\end{tikzpicture}}
\newcommand{\Umu}[1]{\begin{tikzpicture}[every node/.style={sloped,allow upside down},baseline=0ex,place/.style={inner sep =0.2mm,circle,draw=black,fill=black}]
\draw[] (0,0) node[anchor=south] {$x$} -- node {\midarrow} (0.5,0);
\node[place] at (0,0) {};
\node[anchor=south] at (0.25,0) {\tiny $\hat{#1}$};
\end{tikzpicture}}
\newcommand{\umu}[1]{\begin{tikzpicture}[baseline=-0.5ex,place3/.style={inner sep =0.4mm,circle,draw=black,fill=red!90!black},]
\node[place3] at (0,0)  {};
\node[anchor=south] at (0,0) {$x$};
\node[anchor=north] at (0,0) {\tiny $#1$};
\end{tikzpicture}}
\newcommand{\umup}[1]{\begin{tikzpicture}[baseline=-0.5ex]
\shade [ball color=red!90!black] (0,0) circle [radius=0.075];
\node[anchor=south] at (0,0) {$x$};
\node[anchor=north] at (0,0) {\tiny $#1$};
\end{tikzpicture}}
\newcommand{\umuo}[1]{\begin{tikzpicture}[baseline=-0.5ex,place3/.style={inner sep =0.4mm,circle,draw=black,fill=orange},]
\node[place3] at (0,0)  {};
\node[anchor=south] at (0,0) {$x$};
\node[anchor=north] at (0,0) {\tiny $#1$};
\end{tikzpicture}}
\newcommand{\umuop}[1]{\begin{tikzpicture}[baseline=-0.5ex,]
\shade [ball color=orange] (0,0) circle [radius=0.075];
\node[anchor=south] at (0,0) {$x$};
\node[anchor=north] at (0,0) {\tiny $#1$};
\end{tikzpicture}}
\newcommand{\umuoo}[1]{\begin{tikzpicture}[baseline=-0.5ex,place3/.style={inner sep =0.4mm,circle,draw=black,fill=yellow!80!black},]
\node[place3] at (0,0)  {};
\node[anchor=south] at (0,0) {$x$};
\node[anchor=north] at (0,0) {\tiny $#1$};
\end{tikzpicture}}
\newcommand{\umuoop}[1]{\begin{tikzpicture}[baseline=-0.5ex]
\shade [ball color=yellow!80!black] (0,0) circle [radius=0.075];
\node[anchor=south] at (0,0) {$x$};
\node[anchor=north] at (0,0) {\tiny $#1$};
\end{tikzpicture}}
\newcommand{\Umud}[1]{\begin{tikzpicture}[every node/.style={sloped,allow upside down},baseline=0ex,place/.style={inner sep =0.2mm,circle,draw=black,fill=black}]
\draw[] (0.5,0) node[anchor=south west] {$x+\hat{#1}$} -- node {\midarrow} (0,0);
\node[anchor=south] at (0.25,0) {\tiny $\hat{#1}$};
\node[place] at (0.5,0) {};
\end{tikzpicture}}
\newcommand{\Umudm}[1]{\begin{tikzpicture}[every node/.style={sloped,allow upside down},baseline=0ex,place/.style={inner sep =0.2mm,circle,draw=black,fill=black}]
\draw[] (0.5,0) node[anchor=south west] {$x$} -- node {\midarrow} (0,0);
\node[anchor=south] at (0.25,0) {\tiny $\hat{#1}$};
\node[place] at (0.5,0) {};
\end{tikzpicture}}
\newcommand{\EUmu}[2]{\begin{tikzpicture}[baseline=0ex,place3/.style={inner sep =0.4mm,circle,draw=black,fill=green!70!black},]
\node[anchor=south] at (0,0) {$x$};
\draw[] (0,0) -- node {\midarrow} (0.5,0);
\node[anchor=north] at (0,0) {\tiny $#1$};
\node[anchor=south] at (0.25,0) {\tiny $\hat{#2}$};
\node[place3] at (0,0)  {};
\end{tikzpicture}}
\newcommand{\tikzcuboid}[4]{% width, height, depth, scale
\begin{tikzpicture}[scale=#4]
\foreach \x in {0,...,#1}
{
\draw (\x,0,#3) -- (\x,#2,#3);
\draw (\x,#2,#3) -- (\x,#2,0);
}
\foreach \x in {0,...,#2}
{
\draw (#1,\x,#3) -- (#1,\x,0);
\draw (0,\x,#3) -- (#1,\x,#3);
}
\foreach \x in {0,...,#3}
{
\draw (#1,0,\x) -- (#1,#2,\x);
\draw (0,#2,\x) -- (#1,#2,\x);
}
\foreach \x in {0,...,#1}
{
\foreach \y in {0,...,#2}
{
\node[redball] at (\x,\y,#3) {\tiny\phantom{a}};
}
\foreach \y in {0,...,#3}
{
\node[redball] at (\x,#2,\y) {\tiny\phantom{a}};
}
}
\foreach \x in {0,...,#2}
{
\foreach \y in {0,...,#3}
{
\node[redball] at (#1,\x,\y) {\tiny \phantom{a}};
}
}
\end{tikzpicture}
}

\newcommand{\tikzcube}[2]{%lenght, scale
\tikzcuboid{#1}{#1}{#1}{#2}
}
%\geometry{hmargin=2.5cm,vmargin=2cm}
\usepackage{setspace}
%\setstretch{1,5}
\usepackage{pagedecouv}

\usepackage[boxruled,vlined]{algorithm2e}
\providecommand{\SetAlgoLined}{\SetLine}
\usepackage{float}
\floatstyle{plain}
\newfloat{myalgo}{tbhp}{mya}
\newenvironment{Algorithm}[2][tbh]%
{\begin{myalgo}[#1]
\centering
\begin{minipage}{#2}
\begin{algorithm}[H]}%
{\end{algorithm}
\end{minipage}
\end{myalgo}}


\setcounter{secnumdepth}{4}
\setcounter{tocdepth}{1}
\makeatletter
\newcounter {subsubsubsection}[subsubsection]
\renewcommand\thesubsubsubsection{\thesubsubsection .\@alph\c@subsubsubsection}
\newcommand\subsubsubsection{\@startsection{subsubsubsection}{4}{\z@}%
                                     {-3.25ex\@plus -1ex \@minus -.2ex}%
                                     {1.5ex \@plus .2ex}%
                                     {\normalfont\normalsize\bfseries}}
\renewcommand\paragraph{\@startsection{paragraph}{5}{\z@}%
                                    {3.25ex \@plus1ex \@minus.2ex}%
                                    {-1em}%
                                    {\normalfont\normalsize\bfseries}}
\renewcommand\subparagraph{\@startsection{subparagraph}{6}{\parindent}%
                                       {3.25ex \@plus1ex \@minus .2ex}%
                                       {-1em}%
                                      {\normalfont\normalsize\bfseries}}
\newcommand*\l@subsubsubsection{\@dottedtocline{4}{10.0em}{4.1em}}
\renewcommand*\l@paragraph{\@dottedtocline{5}{10em}{5em}}
\renewcommand*\l@subparagraph{\@dottedtocline{6}{12em}{6em}}
\newcommand*{\subsubsubsectionmark}[1]{}
\makeatother

\makeatletter
\def\toclevel@subsubsubsection{4}
\def\toclevel@paragraph{5}
\def\toclevel@subparagraph{6}
\makeatother
\tikzset{
    hyperlink node/.style={
        alias=sourcenode,
        append after command={
            let     \p1 = (sourcenode.north west),
                \p2=(sourcenode.south east),
                \n1={\x2-\x1},
                \n2={\y1-\y2} in
            node [inner sep=0pt, outer sep=0pt,anchor=north west,at=(\p1)] {\hyperlink{#1}{\phantom{\rule{\n1}{\n2}}}}
        }
    }
}
\usepackage{appendix}
%\usepackage{chngcntr}
%\counterwithout{figure}{chapter}


\usepackage{etoolbox}
\usepackage{lipsum}
\AtBeginEnvironment{subappendices}{%
\section*{Appendix}
\addcontentsline{toc}{section}{Appendices}
%\counterwithin{figure}{section}
%\counterwithin{table}{section}
}

%%% Auteurs en gras %%%
\renewbibmacro*{author}{%
\mkbibbold{%
  \ifboolexpr{
    test \ifuseauthor
    and
    not test {\ifnameundef{author}}
  }
    {\printnames{author}%
     \iffieldundef{authortype}
       {}
       {\setunit{\addcomma\space}%
        \usebibmacro{authorstrg}}}
    {}}}
%%% Auteurs en gras %%%
